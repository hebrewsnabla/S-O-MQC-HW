%\documentclass[UTF8]{ctexart} % use larger type; default would be 10pt
\documentclass[a4paper]{article}
\usepackage{../mqc}

\renewcommand\thesection{\Alph{section}}

\title{\textbf{Modern Quantum Chemistry, Szabo \& Ostlund}\\HW}
\author{wsr
\vspace{5pt}\\
}
\date{\today} % Activate to display a given date or no date (if empty),
         % otherwise the current date is printed 

\begin{document}
% \boldmath

\maketitle

\tableofcontents

\newpage

%\setcounter{section}{6}

\section{Integral Evaluation with 1s Primitive Gaussians}

\section{2-Electron Self-consistent-field Program}

\section{Analytic Derivative Methods and Geometry Optimization}
\subsection{Introduction}

\subsection{General Considerations}

\subsection{Analytic Derivatives}

\subsection{Optimization Techniques}

\subsection{Some Optimization Algorithms}
\ex{C.1}
\subex{(a)}
\begin{align}
\vb{H} &= \mqty(\dis\pdv[2]{E}{x} & \dis\pdv{E}{x}{y} \vspace{5pt}\\ \dis\pdv{E}{y}{x} & \dis\pdv[2]{E}{y}) \notag\\
&= \mqty(K & K'' \\ K'' & K')
\end{align}
\begin{align}
\vb{f}(\vb{X}) &= \mqty(\dis\pdv{E}{x} \vspace{5pt}\\ \dis\pdv{E}{y}) \notag\\
&= \mqty(K(x-a) + K''y \\ K'(y-b) + K''x) \notag\\
&= \mqty(K & K'' \\ K'' & K')\mqty(x\\ y) - \mqty(K a \\ K' b) \notag\\
&= \mqty(K & K'' \\ K'' & K')\vb{X} - \mqty(K a \\ K' b)
\end{align}
\begin{align}
\vb{q} &= -\vb{H}^{-1}\vb{f} \notag\\
&= -\vb{X} + \mqty(K & K'' \\ K'' & K')^{-1} \mqty(K a \\ K' b) \notag\\
&= -\vb{X} + \dfrac{1}{KK' - K''^2} \mqty(K' & -K''\\ -K'' & K) \mqty(K a \\ K' b) \notag\\
&= -\vb{X} + \dfrac{1}{KK' - K''^2} \mqty(KK' a - K'K'' b \\ -KK'' a + KK' b) \notag\\
&= -\vb{X} + \dfrac{1}{KK' - K''^2} \mqty(KK' & -K'K'' \\ -KK'' & KK' ) \mqty(a\\ b)
\end{align}

\subex{(b)}
Since $ \vb{q} = \vb{X}_e - \vb{X} $,
\begin{align}
\vb{X}_e &= \dfrac{1}{KK' - K''^2} \mqty(KK' & -K'K'' \\ -KK'' & KK' ) \mqty(a\\ b) \notag\\
&= \dfrac{1}{0.1 - K''^2} \mqty(0.1 & -0.1 K'' \\ -K'' & 0.1 ) \mqty(3\\ 2) %\notag\\
\end{align}

\begin{table}
	\centering
	\begin{tabular}{cc}
		\hline
		$ K'' $ & $ \vb{X}_e = (x_e, y_e) $\\ \hline
		0 & (3.000, 2.000)\\
		0.010 & (2.983, 1.702)\\
		0.030 & (2.967, 1.110)\\ \hline
	\end{tabular}
\end{table}

\ex{C.2}
\begin{align}
\vb{H} 
&= \mqty(K & K'' \\ K'' & K') \notag\\
&= \mqty(1.000 & 0.030\\ 0.030 & 0.100)
\end{align}
\begin{align}
\vb{f} 
&= \mqty(K & K'' \\ K'' & K')\vb{X} - \mqty(K a \\ K' b) \notag\\
&= \mqty(1.000 & 0.030\\ 0.030 & 0.100)\mqty(3.3\\ 1.8) 
- \mqty(1.000\times 3.00\\ 0.100\times 2.00) \notag\\
&= \mqty(0.354\\ 0.079)
\end{align}
\begin{align}
\vb{q} &= -\vb{H}^{-1}\vb{f} \notag\\
&= -\mqty(1.000 & 0.030\\ 0.030 & 0.100)^{-1} \mqty(0.354\\ 0.079) \notag\\
&= \mqty(-0.333, -0.690)
\end{align}
thus
\begin{align}
\vb{X}_e &= \vb{q} + \vb{X} \notag\\
&= \mqty(2.967, 1.110)
\end{align}
which agrees with the result in Ex C.1(b).

\ex{C.3}
A program is written to solve this problem, which is \code{C-3.py}. 

For example, run the program by
\code{python C-3.py 0.03}, and the Nelder-Mead optimization steps will be printed for $ K'' = 0.03 $.

\ex{C.4}
A program is written to solve this problem, which is \code{C-4.py}. 

For example, run the program by
\code{python C-4.py}, and the MS optimization steps will be printed.

\subsection{Transition States}

\subsection{Constrained Variation}

\section{Molecular Integrals for \ce{H_2} as a Function of Bond Length}





\end{document}